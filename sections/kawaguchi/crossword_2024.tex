%%%%%%%%%%%%%%%%%%%%%%%%%%%%%%%%%%%%%%%%%%%%%%%%%%%%%%%%%%%%%%%%%%%%%%
% \documentclass[a4paper,10pt]{jarticle}
% \usepackage[dvipdfmx]{graphicx}
% \usepackage{amsmath, amssymb}
% \usepackage{latexsym}
% \usepackage{multirow}
% \usepackage{here}
% \usepackage{url}
% \usepackage{bm}
% \usepackage{longtable}
% \usepackage{dcolumn}
% \usepackage{ascmac}
% \setlength{\textwidth}{165mm} %165mm-marginparwidth
% \setlength{\marginparwidth}{40mm}
% \setlength{\textheight}{240mm}
% \setlength{\topmargin}{-20mm}
% \setlength{\oddsidemargin}{-3.5mm}
%
% \def\vector#1{\mbox{\boldmath $#1$}}
% \newcommand{\AmSLaTeX}{%
%  $\mathcal A$\lower.4ex\hbox{$\!\mathcal M\!$}$\mathcal S$-\LaTeX}
% \newcommand{\PS}{{\scshape Post\-Script}}
% \def\BibTeX{{\rmfamily B\kern-.05em{\scshape i\kern-.025em b}\kern-.08em
%  T\kern-.1667em\lower.7ex\hbox{E}\kern-.125em X}}
% \newcommand{\pderiv}[2]{{\partial#1\over\partial#2}}
% \newcommand{\deriv}[2]{{{\rm d}#1\over{\rm d}#2}}
% \newcommand{\dderiv}[2]{{{\rm d}^2#1\over{\rm d}#2^2}}
% \newcommand{\DeLta}{{\mit\Delta}}
% \renewcommand{\d}{{\rm d}}
% \def\wcaption#1{\caption[]{\parbox[t]{100mm}{#1}}}
% \def\rm#1{\mathrm{#1}}
% \def\tempC{^\circ \rm{C}}
%
%
% \makeatletter
% %\def\section{\@startsection {section}{1}{\z@}{-3.5ex plus -1ex minus % -.2ex}{2.3ex plus .2ex}{\Large\bf}}
% \def\section{\@startsection {section}{1}{\z@}{-3.5ex plus -1ex minus
% -.2ex}{2.3ex plus .2ex}{\normalsize\bf}}
% \makeatother
%
% \makeatletter
% \def\subsection{\@startsection {subsection}{1}{\z@}{-3.5ex plus -1ex minus
% -.2ex}{2.3ex plus .2ex}{\normalsize\bf}}
% \makeatother
%
% \makeatletter
% \def\@seccntformat#1{\@ifundefined{#1@cntformat}%
%    {\csname the#1\endcsname\quad}%      default
%    {\csname #1@cntformat\endcsname}%    enable indiv idual control
% }
% \makeatother
%
%%%%%%%%%%%%%%%%%%%%%%%%%%%%%%%%%%%%%%%%%%%%%%%%%%%%%%%%%%%%%%%%%%%%%%
%! TEX root = ../main.tex
\documentclass[../main]{subfiles}

\begin{document}
% \rightline{4年(OB) 川口 美玲} % 学年と名前(ハンドルネームでも可)
{
\small
\vskip\baselineskip
\vskip\baselineskip
\vskip\baselineskip
☆は難易度を表しています!
☆易$\Longleftrightarrow$難☆☆☆\\
※小さい文字(ッ、ョ、ャなど)は大きい文字としてください。
\vskip\baselineskip
\begin{itembox}[l]{たてのカギ}

1.  29歳で最年少飛行士に正式認定された宇宙飛行士、米田〇〇さん。☆\\
2.  おもに南半球で見ることができる、航海用の方位磁石の形を表した星座〇〇〇〇〇座。☆☆\\
3.  今年の8月をもって月面での運用を終了した小型月着陸実証機。☆☆\\
4.  トランジット法によって宇宙から太陽系外惑星を探査するための人工衛星〇〇衛星。☆☆☆\\
6.  天王星と海王星はそっくりの双子の惑星と言われたりします。よく似ていること。☆\\
7.  原子番号、質量数が同じでエネルギー状態や半減期が異なる原子核〇〇〇〇〇体。☆☆\\
8.  航空宇宙を英訳すると〇〇〇スペース。☆☆\\
12. おとめ座のα星。ラテン語で麦の穂。☆\\
15. 空気中の水蒸気がまとまった小さい水のつぶの集まり。☆\\

\end{itembox}
\vskip\baselineskip
\vskip\baselineskip
\begin{itembox}[l]{よこのカギ}
1.  皆様はご覧になりましたか?2023年1月に発見された紫金山・〇〇〇〇彗星。☆\\
4.  表面温度が〇〇温であると、星は赤っぽく見えます。☆\\
5.  太陽を除いて地球上から見える最も明るい構成。おおいぬ座α星。☆\\
7.  今年の2月~9月に新星爆発が起こると予想されていた〇〇〇〇座T。しかし爆発の期待はまだ捨ててはいけませんよ!☆☆\\
9.  茨城県にあります。〇〇〇宇宙センター。☆\\
10. 太陽の表面で爆発がおこったもの。人工衛星の故障の原因になったりします。☆☆\\
11. 電気通信大学天文部の〇〇〇〇のユーザー名はuec$\_$astro。フォローお願いします!(写真がメインのSNS)☆\\
13. メキシコ移民の貧しい家庭で育った〇〇・ヘルナンデス宇宙飛行士。彼の宇宙飛行士になるまでを描いた映画『ミリオン・マイルズ・アウェイ~遠き宇宙への旅路~』がアマゾンプライムで配信されています。☆☆☆\\
14. フランス・ピレネー山脈を見渡せる標高2800mあまり、130年の歴史を誇る〇〇〇・デュ・ミディ天文台。いつか行ってみたい!!☆☆☆\\
16. 4つの星が作る小さなひし形が目印の星座。夏の大三角の東側にある〇〇〇座。☆☆\\
17. 太陽系最大の惑星、〇〇星。☆\\
\end{itembox}
\vskip\baselineskip
\rightline{挑戦してくださってありがとうございました!}
\vskip\baselineskip
参考文献\\
天文学辞典, \url{https://astro-dic.jp/}\\
天文宇宙検定公式テキスト4級, 3級
}


\end{document}
