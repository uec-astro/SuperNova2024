%! TEX root = ../main.tex
\documentclass[../../super_nova_2024]{subfiles}

% ローカル下書き用
% \documentclass{supernova_pre}
% このクラスの中に大体のパッケージは入ってるので基本何でもかけるはず
% 追加したいパッケージがあればここに記入


\begin{document}

\chapter{彗星撮影にチャレンジ!} % タイトル
\rightline{M1 森山 陽介} % 学年と名前(ハンドルネームでも可)


\section{最近彗星がアツい!}

天文部員として活動をはじめて4年目になります。もともと写真部だったのもあって星景写真を主に撮っていましたが、天文部に入ってからは望遠鏡と赤道儀を使った天体写真に挑戦しています。星雲・星団の撮影のため、貯めたアルバイト代で自前の望遠鏡と赤道儀が揃ってきた去年から今年にかけては、明るい彗星も頻繁に到来するチャンスの年でした。本稿では、観測・撮影に挑戦したZTF彗星 (C/2022 E3 (ZTF))、ポンス・ブルックス彗星 (12P/Pons-Brooks)、紫金山・アトラス彗星 (C/2023 A3 (Tsuchinshan-ATLAS))についてお伝えします。

\section{彗星について}

まず最初に、彗星についておさらしいましょう。彗星は惑星や準惑星、小惑星と並ぶ太陽系を構成する天体です。
構成成分の8割を占める水と、二酸化炭素、一酸化炭素やその他のガスと微量の塵から成る氷の塊で、よく「砂で汚れた雪玉」と表現されます。太陽に近づいたときにその熱で本体から融けたガスや塵が輝くことで「コマ」と呼ばれる淡い光が本体を包むように形成されます。また、彗星は「ほうき星」とも呼ばれるように、長い尾を引く姿でも有名です。融けたガスと塵によって2種類の尾が作られます。ひとつは、太陽紫外線を受けてイオンと電子に電離したガスが太陽風を受けて作る「イオンの尾」、もうひとつに、塵が太陽光の圧力「光圧」によって広がり太陽光を散乱して作る「ダストの尾」です。
この塵の一部は、彗星の軌道を周回し続け、その軌道に地球がぶつかる時に流星群として観測されます。先月のオリオン座流星群の母天体はあのハレー彗星 (1P/Halley)です。

黄道面と呼ばれる平面に沿ってほぼ円形の楕円軌道で公転する惑星と違い、彗星の軌道は細長い楕円軌道や、放物線・双曲線軌道を描きます。放物線・双曲線軌道のように一度だけ太陽に近づき二度と戻ってこない彗星や、楕円軌道で公転周期が200年より大きい彗星は「長周期彗星」と呼ばれ、200年以内の彗星は「短周期彗星」と呼ばれています。彗星などの天体は、発見されてから仮符号がつけられます。彗星ではCometを表す``C/''が接頭語として付加されます。この中で、短周期彗星と2回以上の近日点を持つ長周期彗星と判明した場合には``P/''の接頭語になります。また、消滅した彗星には``D/''が、短期間の観測で不確定な彗星には``X/''が振られます。この接頭語に続いて、「発見された年」と「発見された月を示すアルファベット」 (\tabref{alphabet})、「期間内の発見順の番号」が続きます。また、この後ろには括弧書きで発見者・観測プロジェクトの名前が表記されます。
先月地球に接近した紫金山・アトラス彗星は``C/2023 A3 (Tsuchinshan-ATLAS)''という符号が振られており、「2023年1月前半の3番目に紫金山天文台とATLAS天文台によって発見された彗星」ということになります。

\begin{table}
    \centering
    \caption{発見された月を示すアルファベットの対照表。Iを除くA--Yのアルファベットが各月の前半後半それぞれに順に対応している。}
    \label{tab:alphabet}
    \begin{tabular}{c|cccccccccccc}\hline
        月 & 1月 & 2月 & 3月 & 4月 & 5月 & 6月 & 7月 & 8月 & 9月 & 10月 & 11月 & 12月 \\
        \hline
        前半(1--15日) & A & C & E & G & J & L & N & P & R & T & V & X \\
        後半(16日--) & B & D & F & H & K & M & O & Q & S & U & W & Y \\
        \hline
    \end{tabular}
\end{table}


\section{ZTF彗星 (C/2022 E3 (ZTF))}

私が天文部に入ってから初めて観測した彗星がZTF彗星でした。2022年3月2日にアメリカのZwicky Transient Facility (ZTF)による観測で発見され、近日点は2023年1月13日、2023年2月2日に近地点を迎えました。SNSでこの彗星の接近を知った当時学域3年生の私たちは、2023年2月1日から2日にかけての晩にいつもの神代植物公園自由広場に集まり観測を試みました。明るい惑星や、いつも同じ場所にいるメシエ天体などとは異なり、彗星は暗く日によって大きく移動するため、前日からの予習が効きません。そこで当日手動で望遠鏡と赤道儀を使って探し出すのは難しいだろうと、ひとまずカメラと望遠レンズと三脚だけで存在だけでも確認しようとしました。当日のZTF彗星は、ぎょしゃ座の一等星カペラと北斗七星の柄杓の先端にあるドゥーべ、北極星ポラリスを結んだ三角形の内心のような位置(カペラ寄り)に見つけることができました。\figref{ZTF}は当時撮影した一枚です。画像中央にボヤッとした星が写っているのがわかると思います。これがZTF彗星です。

\section{ポンス・ブルックス彗星 (12P/Pons-Brooks)}



\section{紫金山・アトラス彗星 (C/2023 A3 (Tsuchinshan-ATLAS))}



\section{さいごに}



\end{document}