%! TEX root = ../main.tex
\documentclass[../../super_nova_2024]{subfiles}

\begin{document}
\chapter{番外編〜天体観測への持ち物〜}
\rightline{M1 中原佑之助} % 学年と名前(ハンドルネームでも可)

\begin{verse}
  {\slshape
  午前2時踏切に〜望遠鏡を担いでった〜〜
  }
  {\footnotesize(天体観測 BUMP OF CHICKEN)}
\end{verse}



天体観測には望遠鏡を担いでいくというイメージが多いのではないでしょうか、しかしながら、望遠鏡を担いでいくと、それを支える大きくて重い三脚と赤道儀も担いでいかなければいけません。いきなりそんなガチガチの天体観測は嫌だ!\footnote{というかそんな道具は持ってない!}という人に向けて天体観測する時の持ち物を紹介しようと思います!

\section{必要なもの}
\subsection{防寒具}
真夏を除いて、夜は寒いです。寒い夜に星を見ながらじっとしているとそれはそれは想像を絶する寒さです。天体観測初心者の時は、夜の寒さを舐めてかかり、痛い目に会いました。また、寒さに耐えられなくなり、車や建物の中に戻り、一度快適さを味わうともう外には戻って来られません。夜通し天体観測を行った者しか味わえない特別な朝日を見るために、防寒具は欠かせません。

では、具体的にどのくらいの防寒具を用意しているか紹介します。まず衣類ですが、僕は9月末や3月中旬の合宿では上はヒートテックの上に、長袖Tシャツ2枚、パーカー、ダウンジャケット、下はヒートテックにスウェット、ジーンズ、風を通しにくい素材のズボンを重ね着し、靴下は2枚重ねです。また、パーカーのフードをかぶる、マフラーやネックウォーマーで首を温めることも大事です。

次に衣類以外です。体感ですが、カイロはあまり役に立ちません。持っていかなくていいです。寝袋があると、寝転がって天体観測できるのであると良いです。

\section{あると良いもの}
必要なものを考えたら防寒具だけでした、身一つあればできる趣味っていいですね!

\subsection{折りたたみ椅子}
ブルーシート敷いて寝袋に入るのもいいんですけど、地面に接してると寒いです。立ちっぱなしで過ごすには夜は長いすぎるので寝転がるか座るかしたい時に、折りたたみ椅子はおすすめです。
カメラで撮影する場合も、寝転がっているとカメラや望遠鏡の操作ができませんが、椅子に座っていれば簡単な操作ぐらいならできます。

座ったまま星を見上げたいので、深く座ることができる背もたれ付きの椅子が良いでしょう。通販でも3000から5000円あれば購入できるのでかなりおすすめです。

\subsection{食べ物、カフェイン入りの飲み物}
19時に夕飯を済ませたとすると、翌朝の朝ごはんまで10時間以上は空くことになります。お腹が空くことは当然として、体にエネルギーを入れてあげないと、どんどん寒くなります。そこでチョコレートやクッキーなど簡単に食べられるものを用意しておくと良いでしょう。みんなで食べ物をシェアするのも夜のピクニックみたいで楽しいので、ぜひおすすめのお菓子を持ち寄ってください。

徹夜をしようとするとどうしても眠くなるので、コーヒーやエナジードリンクなどでカフェインを摂るのも良いでしょう。個人的にはチョコレートやクッキーと一緒にピクニック気分を楽しみたいので、コーヒーを用意することが多いです。また、朝の景色を眺めながらのコーヒーは格別なので、最高の朝を過ごしてください。

\subsection{双眼鏡}
\begin{verse}
  {\slshape
  見えないものを見ようとして望遠鏡を覗き込んだ〜
  }
  {\footnotesize(天体観測 BUMP OF CHICKEN)}
\end{verse}


ここまで、お菓子だのコーヒーだの全然星を見ることと関係ないものでした。ようやく天体観測っぽい道具の登場です!

双眼鏡で夜空を見ると、星一つ一つを鮮明に、明るく見ることができます。また、プレアデス星団などの明るめの星団や銀河も双眼鏡で観察することができます。肉眼では見えない天体を見たり、小さな天体を大きく見たりするだけでなく、その位置を手軽に確認するのにも役立ちます。

10月の中旬ごろの紫金山・アトラス彗星を見る時にも、部の双眼鏡を使うことでその尾をはっきりと確認することができました。

\subsection{天気予報}
\begin{verse}
  {\slshape
    ベルトに結んだラジオ〜雨は降らないらしい

    Oh yeah ah

    Ah ah ah ah yeah yeah

  }
  {\footnotesize(天体観測 BUMP OF CHICKEN)}
\end{verse}
天体観測する前に天気はしっかり確認しておきましょう。

ラジオでじゃないですよ、星を見るのに良い「晴れ」は普通の天気予報の「晴れ」ではなく、快晴レベルである必要があります。SCW\footnote{SCW天気予報, \url{https://supercweather.com}}や星空指数(検索すると出てくる)などで、「晴れ」なのかしっかり確認しましょう。


\subsection{友達}
\begin{verse}
  {\slshape
  今というほうき星

  君と2人追いかけている

  Oh yeah ah

  Ah ah ah ah yeah yeah

  }
  {\footnotesize(天体観測 BUMP OF CHICKEN)}
\end{verse}

夜は1人だと心細いし、退屈になってくるので、是非友達と出かけることをお勧めします。

\end{document}
