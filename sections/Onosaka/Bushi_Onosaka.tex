%! TEX root = ../main.tex
\documentclass[../main]{subfiles}



\begin{document}
\chapter{メシエ天体の欠番について} %jsarticleだとchapterがうまく動かないけど、部誌の環境だと大丈夫だと思うのでそのまま置いときます不都合あったら教えてください

\rightline{3年 小野坂 潤一郎}

\subsection*{星の写真について}
私たち天文部はよく星の写真を撮ります。
星空と風景を一緒に写した星景写真や、景色は写さないで星空を主役として写した星野写真を合宿に行った時によく撮ります。
そして私たちがよく撮る写真にはもう1つ、天体写真というのがあります。
天体写真とは天体望遠鏡などを用いて、惑星、星雲、星団、銀河など、特定の天体を大きく写した写真のことです。

\subsection*{メシエ天体とは}
私たちが天体写真としてよく撮る天体にメシエ天体というものがあります。
メシエ天体というのはフランスの天文家であるシャルル・メシエが作った星雲、星団、銀河などのリストに登録されている天体のことです。
当時、メシエは彗星と紛らわしい天体をリストアップして彗星探索の役に立てようとしていました。
そのリストがメシエカタログです。\\
メシエカタログに登録されている天体には番号(メシエ番号)が割り当てられていて、番号は1番から110番まであります。
メシエ天体はそのメシエ番号の前にMをつけてM〇と表記されます。
例えばメシエ天体として有名なアンドロメダ銀河はM31、オリオン大星雲はM42、プレアデス星団はM45と表されます。
これらの画像をこの下に貼り付けたいのですが、私はちゃんとした天体写真を撮ったことがないので無理でした…。
先輩方は綺麗なメシエ天体を撮っているので気になる方はぜひ展示を見てください。

\subsection*{メシエカタログの欠番}
先ほどメシエ天体の番号は1番から110番まであると言いましたが、実はM40、M91、M102は欠番とされることが多いです。
なぜこれらの番号は欠番となっているのでしょうか?\\
M40は実は二重星で「彗星と紛らわしい星雲、星団、銀河」ではないのです。
しかし、それを星雲と見誤って報告されたのをメシエがM40としてメシエカタログに加えたといわれています。
M40はメシエカタログの「彗星と紛らわしい天体」という趣旨に沿わないため、メシエカタログから外されることが多いです。
このような理由でM40は欠番とされることが多いです。
\\
M91はメシエカタログに登録されてはいるが、場所がよくわからない天体です。
そのため存在が確認できずに欠番とされることが多いです。
\\
M102も同様に存在が確認できずに欠番とされることが多いです。
M102が追加された当時にも「M102はM101を見間違えたものだ」と否定されています。
\\
このようにメシエ天体には欠番が存在するのです。
\\
しかしこれらの欠番に具体的な天体を割り当てて補完しようとする試みもあります。
M40はおおぐま座の二重星、M91はかみのけ座の銀河、M102はりゅう座の銀河とする説はよく知られています。
この他にもいくつか説はあります。

\subsection*{終わり}
天文部もいくつかのメシエ天体の天体写真を撮っています。
展示されている天体写真を見るとよりメシエ天体を楽しめると思います。
また、メシエ天体の位置がわかる星図も展示しているのでそれも一緒に見てみてください!



\subsubsection*{参考文献}
AstroArts."メシエ天体ガイド".AstroArts.2014.\url{https://www.astroarts.co.jp/alacarte/messier/index-j.shtml}.(参照 2024-10-13)


\end{document}