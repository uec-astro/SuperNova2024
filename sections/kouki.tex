%! TEX root = ../supernova_20yy.tex
\documentclass[../super_nova_20yy]{subfiles}

\begin{document}
\chapter{編集後記}
\vspace{2\zw}

ここまで読んでくださり、ありがとうございます。これを読めているということは、今年も無事に部誌を発刊できたということでしょう。2年前の第72回調布祭出展の際、創部当初発刊されていた部誌「スーパーノヴァ(すーぱーのゔぁ)」をコロナ禍からの再出発の気持ちを込めて、新生\SuperNova として発刊を再開させました。今年で3回目の調布祭を迎え、再発刊から通算4回目の刊行になります。今号も、前号を超える分厚い記事が寄せられました。寄稿してくれた部員の皆様には感謝しています。

これまで3回の部誌刊行の中で、締切から頒布までの期間が短いことから製本作業がタイトになっていました。
これと、日々課題に追われる電通大生活を鑑みて、夏休み終了後1週間の比較的負担の少ない週に締切を設定しました。しかし、締切当日になっても記事はなかなか集まらず、結局例年と同じように2回の締切延長を経ることになりました。
締切に間に合わない要因としては、課題に忙しいことはもちろんですが、部誌制作に使っている\LaTeX に慣れておらず抵抗感がある\footnote{電通大生なら\LaTeX は必修なので余裕だと思ってましたがそうではないようです。}、部誌に何を書けば良いのかわからないなどがあるようでした。
部誌の発行は、天文部の活動を記録しておくことに加え、部員自身の天文に関する見識を深め、活動を深めることができる機会です。
私も、これまで4回の部誌刊行に携わりましたが、そろそろ次の世代へ繋げていくことも急務であると考えています。
安定的な部誌の刊行に向けて、\LaTeX 講習会や天文に関する勉強会を定期的に行うなど、年間スケジュールを見据えた長期的な視点で、部誌編集が行えるような体制を準備していけたらと思います。


\rightline{部誌編集長 修士1年 森山 陽介~\includegraphics[height=2\zw]{figures/uec_astro_logo.png}}
\end{document}
